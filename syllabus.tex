% Options for packages loaded elsewhere
\PassOptionsToPackage{unicode}{hyperref}
\PassOptionsToPackage{hyphens}{url}
\PassOptionsToPackage{dvipsnames,svgnames,x11names}{xcolor}
%
\documentclass[
  letterpaper,
  DIV=11,
  numbers=noendperiod]{scrartcl}

\usepackage{amsmath,amssymb}
\usepackage{iftex}
\ifPDFTeX
  \usepackage[T1]{fontenc}
  \usepackage[utf8]{inputenc}
  \usepackage{textcomp} % provide euro and other symbols
\else % if luatex or xetex
  \usepackage{unicode-math}
  \defaultfontfeatures{Scale=MatchLowercase}
  \defaultfontfeatures[\rmfamily]{Ligatures=TeX,Scale=1}
\fi
\usepackage{lmodern}
\ifPDFTeX\else  
    % xetex/luatex font selection
\fi
% Use upquote if available, for straight quotes in verbatim environments
\IfFileExists{upquote.sty}{\usepackage{upquote}}{}
\IfFileExists{microtype.sty}{% use microtype if available
  \usepackage[]{microtype}
  \UseMicrotypeSet[protrusion]{basicmath} % disable protrusion for tt fonts
}{}
\makeatletter
\@ifundefined{KOMAClassName}{% if non-KOMA class
  \IfFileExists{parskip.sty}{%
    \usepackage{parskip}
  }{% else
    \setlength{\parindent}{0pt}
    \setlength{\parskip}{6pt plus 2pt minus 1pt}}
}{% if KOMA class
  \KOMAoptions{parskip=half}}
\makeatother
\usepackage{xcolor}
\setlength{\emergencystretch}{3em} % prevent overfull lines
\setcounter{secnumdepth}{-\maxdimen} % remove section numbering
% Make \paragraph and \subparagraph free-standing
\makeatletter
\ifx\paragraph\undefined\else
  \let\oldparagraph\paragraph
  \renewcommand{\paragraph}{
    \@ifstar
      \xxxParagraphStar
      \xxxParagraphNoStar
  }
  \newcommand{\xxxParagraphStar}[1]{\oldparagraph*{#1}\mbox{}}
  \newcommand{\xxxParagraphNoStar}[1]{\oldparagraph{#1}\mbox{}}
\fi
\ifx\subparagraph\undefined\else
  \let\oldsubparagraph\subparagraph
  \renewcommand{\subparagraph}{
    \@ifstar
      \xxxSubParagraphStar
      \xxxSubParagraphNoStar
  }
  \newcommand{\xxxSubParagraphStar}[1]{\oldsubparagraph*{#1}\mbox{}}
  \newcommand{\xxxSubParagraphNoStar}[1]{\oldsubparagraph{#1}\mbox{}}
\fi
\makeatother


\providecommand{\tightlist}{%
  \setlength{\itemsep}{0pt}\setlength{\parskip}{0pt}}\usepackage{longtable,booktabs,array}
\usepackage{calc} % for calculating minipage widths
% Correct order of tables after \paragraph or \subparagraph
\usepackage{etoolbox}
\makeatletter
\patchcmd\longtable{\par}{\if@noskipsec\mbox{}\fi\par}{}{}
\makeatother
% Allow footnotes in longtable head/foot
\IfFileExists{footnotehyper.sty}{\usepackage{footnotehyper}}{\usepackage{footnote}}
\makesavenoteenv{longtable}
\usepackage{graphicx}
\makeatletter
\def\maxwidth{\ifdim\Gin@nat@width>\linewidth\linewidth\else\Gin@nat@width\fi}
\def\maxheight{\ifdim\Gin@nat@height>\textheight\textheight\else\Gin@nat@height\fi}
\makeatother
% Scale images if necessary, so that they will not overflow the page
% margins by default, and it is still possible to overwrite the defaults
% using explicit options in \includegraphics[width, height, ...]{}
\setkeys{Gin}{width=\maxwidth,height=\maxheight,keepaspectratio}
% Set default figure placement to htbp
\makeatletter
\def\fps@figure{htbp}
\makeatother

\KOMAoption{captions}{tableheading}
\makeatletter
\@ifpackageloaded{caption}{}{\usepackage{caption}}
\AtBeginDocument{%
\ifdefined\contentsname
  \renewcommand*\contentsname{Table of contents}
\else
  \newcommand\contentsname{Table of contents}
\fi
\ifdefined\listfigurename
  \renewcommand*\listfigurename{List of Figures}
\else
  \newcommand\listfigurename{List of Figures}
\fi
\ifdefined\listtablename
  \renewcommand*\listtablename{List of Tables}
\else
  \newcommand\listtablename{List of Tables}
\fi
\ifdefined\figurename
  \renewcommand*\figurename{Figure}
\else
  \newcommand\figurename{Figure}
\fi
\ifdefined\tablename
  \renewcommand*\tablename{Table}
\else
  \newcommand\tablename{Table}
\fi
}
\@ifpackageloaded{float}{}{\usepackage{float}}
\floatstyle{ruled}
\@ifundefined{c@chapter}{\newfloat{codelisting}{h}{lop}}{\newfloat{codelisting}{h}{lop}[chapter]}
\floatname{codelisting}{Listing}
\newcommand*\listoflistings{\listof{codelisting}{List of Listings}}
\makeatother
\makeatletter
\makeatother
\makeatletter
\@ifpackageloaded{caption}{}{\usepackage{caption}}
\@ifpackageloaded{subcaption}{}{\usepackage{subcaption}}
\makeatother

\ifLuaTeX
  \usepackage{selnolig}  % disable illegal ligatures
\fi
\usepackage{bookmark}

\IfFileExists{xurl.sty}{\usepackage{xurl}}{} % add URL line breaks if available
\urlstyle{same} % disable monospaced font for URLs
\hypersetup{
  pdftitle={Syllabus},
  colorlinks=true,
  linkcolor={blue},
  filecolor={Maroon},
  citecolor={Blue},
  urlcolor={Blue},
  pdfcreator={LaTeX via pandoc}}


\title{Syllabus}
\author{}
\date{}

\begin{document}
\maketitle


\subsection{Course info}\label{course-info}

\subsubsection{Class meetings}\label{class-meetings}

\begin{longtable}[]{@{}
  >{\raggedright\arraybackslash}p{(\columnwidth - 4\tabcolsep) * \real{0.2063}}
  >{\raggedright\arraybackslash}p{(\columnwidth - 4\tabcolsep) * \real{0.4603}}
  >{\raggedright\arraybackslash}p{(\columnwidth - 4\tabcolsep) * \real{0.3333}}@{}}
\toprule\noalign{}
\endhead
\bottomrule\noalign{}
\endlastfoot
\textbf{Lecture} & MWF 3:00 - 3:50pm & Cruzen-Murray Library (CML)
208 \\
\textbf{Homework Lab} & W 4:00 - 4:50pm (Tentative) & Cruzen-Murray
Library (CML) 208 \\
\end{longtable}

\begin{longtable}[]{@{}ll@{}}
\toprule\noalign{}
Office hours (Boone 126B) & \\
\midrule\noalign{}
\endhead
\bottomrule\noalign{}
\endlastfoot
TBD & TBD \\
\emph{or by appointment.} & \\
\end{longtable}

Office hours are also available by appointment, just email me!

\subsubsection{Instructor Information}\label{instructor-information}

\begin{itemize}
\tightlist
\item
  \textbf{Instructor}: Professor Eric Friedlander
\item
  \textbf{Office}: Boone 126B
\item
  \textbf{Email}:
  \href{mailto:efriedlander@collegeofidaho.edu}{\nolinkurl{efriedlander@collegeofidaho.edu}}
\end{itemize}

\subsection{Course Learning
Objectives}\label{course-learning-objectives}

By the end of the semester, you will be able to\ldots{}

\begin{itemize}
\item
  tackle predictive modeling problems arising from real data.
\item
  use R to fit and evaluate machine learning models.
\item
  assess whether a proposed model is appropriate and describe its
  limitations.
\item
  use Quarto to write reproducible reports and GitHub for version
  control and collaboration.
\item
  effectively communicate results results through writing and oral
  presentations.
\end{itemize}

\subsection{Course community}\label{course-community}

\subsubsection{College of Idaho Honor
Code}\label{college-of-idaho-honor-code}

\begin{quote}
The College of Idaho maintains that academic honesty and integrity are
essential values in the educational process. Operating under an Honor
Code philosophy, the College expects conduct rooted in honesty,
integrity, and understanding, allowing members of a diverse student body
to live together and interact and learn from one another in ways that
protect both personal freedom and community standards. Violations of
academic honesty are addressed primarily by the instructor and may be
referred to the
\href{https://collegeofidaho.smartcatalogiq.com/en/current/Undergraduate-Catalog/Policies-and-Procedures/Academic-Misconduct}{Student
Judicial Board}.
\end{quote}

By participating in this course, you are agreeing that all your work and
conduct will be in accordance with the College of Idaho Honor Code.

\subsubsection{Disability Accommodation
Statement}\label{disability-accommodation-statement}

The College of Idaho seeks to provide an educational environment that is
accessible to the needs of students with disabilities. The College
provides reasonable services to enrolled students who have a documented
permanent or temporary physical, psychological, learning, intellectual,
or sensory disability that qualifies the student for accommodations
under the Americans with Disabilities Act or section 504 of the
Rehabilitation Act of 1973. If you have, or think you may have, a
disability that impacts your performance as a student in this class, you
are encouraged to arrange support services and/or accommodations through
the Department of Accessibility and Learning Excellence located in
McCain 201B and available via email at
\href{mailto:accessibility@collegeofidaho.edu}{\nolinkurl{accessibility@collegeofidaho.edu}}.
Reasonable academic accommodations may be provided to students who
submit appropriate and current documentation of their disability.
Accommodations can be arranged only through this process and are not
retroactively applied. More information can be found on the DALE webpage
(\url{https://www.collegeofidaho.edu/accessibility}).

\subsubsection{Communication}\label{communication}

All lecture notes, assignment instructions, an up-to-date schedule, and
other course materials may be found on the course website,
\href{https://mat427sp25.netlify.app}{mat427fa25.netlify.app}.

Periodic announcements will be sent via email and will also be available
through Canvas and grades will be stored in the Canvas gradebook. Please
check your email regularly to ensure you have the latest announcements
for the course.

\subsubsection{In class agreements}\label{in-class-agreements}

If we discuss/agree to something in class or office hours which requires
action from me (e.g.~``you may turn in your homework late due to a
sporting event''), you MUST send me a follow-up message. If you don't, I
will almost certainly forget, and our agreement will be considered null
and void.

\subsubsection{Getting help in the
course}\label{getting-help-in-the-course}

\begin{itemize}
\tightlist
\item
  If you have a question during lecture or lab, feel free to ask it!
  There are likely other students with the same question, so by asking
  you will create a learning opportunity for everyone.
\item
  I am here to help you be successful in the course. You are encouraged
  to attend \emph{office hours} and the \emph{homework lab} to ask
  questions about the course content and assignments. Many questions are
  most effectively answered as you discuss them with others, so office
  hours are a valuable resource. You are encouraged to use them!
\item
  Outside of class and office hours, any general questions about course
  content or assignments can be emailed to me.
\end{itemize}

\subsubsection{Email}\label{email}

If you have questions about assignment extensions or accommodations,
please email
\href{mailto:efriedlander@collegeofidaho.edu}{\nolinkurl{efriedlander@collegeofidaho.edu}}.
Please see \hyperref[late-work-policy]{Late work policy} for more
information. \textbf{If you email me about an error please include a
screenshot of the error and the code causing the error.} Barring
extenuating circumstances, I will respond to MAT 427 emails within 48
hours Monday - Friday. Response time may be slower for emails sent
Friday evening - Sunday.

Check out the \href{support.qmd}{Support} page for more resources.

\subsection{Textbook}\label{textbook}

The official textbook for this course is:

\begin{itemize}
\item
  An Introduction to Statistical Learning with Applications in R by
  Gareth James, Daniela Witten, Trevor Hatie, and Robert Tibshirani

  \begin{itemize}
  \item
    Colloquiually referred to as ``ISLR'', it is considered one of the
    bibles of machine learning
  \item
    It's free!
  \end{itemize}
\end{itemize}

\subsection{Assignments}\label{assignments}

You will be assessed based on five components: homework, job
applications, job interviews, a hack-a-thon, and project.

\subsubsection{Homework}\label{homework}

In homework, you will apply what you've learned during lecture to
complete data analysis tasks. Homework will completed in teams of three,
must be typed up using Quarto, and submitted as .qmd and .pdf files in
via GitHub.

\subsubsection{Job Applications \& Job
Interviews}\label{job-applications-job-interviews}

During this course you will apply to two ``jobs''. I will generate the
job advertisements including real companies and base the job description
on the course content and similar job advertisements that I get from
online or professional collaborators. Each job application will have
three components:

\begin{enumerate}
\def\labelenumi{\arabic{enumi}.}
\tightlist
\item
  A cover letter.
\item
  A resume.
\item
  A portfolio.
\end{enumerate}

All three of these should be tailored to the job description and the
company to which you are applying. Your portfolio will consist of
self-contained data analyses of your choosing. The most straight forward
method of creating th to repurpose your homeworks, converting them from
a format in which you are responding to exercises to something where you
are telling a narrative and demonstrating that you meet the job
criteria. To create your portfolio, you will be required to create a
website. More details on this will be given during the semester, however
the idea of this project is that you will be able to use the things you
general when you are applying for jobs.

After you submit your job applications, you will be invited to schedule
a one-hour long job interview. \textbf{It is your job to schedule your
job interview with Dr.~Friedlander.} Each job interview will have three
portions. The first, lasting 10-15 minutes, will include typical
questions that apply to almost any job interview (e.g.~``What are your
biggest strengths and weaknesses''). The second, lasting 20-30 minutes,
will include questions about the portfolio you submitted and your
understanding of the required skills described in the job advertising.
The third section will mimic what is called a ``case interview''. Case
interviews are extremely common for many jobs, especially those
requiring quantitative or computational skills, and can be intimidating.
During the case interview portion, you will be presented with a ``case
study'' and asked questions on how you would go about approaching it.
The cases themselves will be designed so that they can be solved using
the content from class. The goal of this whole exercise is to assess
your knowledge of the course content in a way that is authentic while
also preparing you to get a job.

\subsubsection{Hack-a-thon}\label{hack-a-thon}

At some point in the semester we will participate in a ``Hack-a-thon''
as a class. Namely, you will be given a short period of time (1-3 days)
to build a model and make a set of predictions. After the competition is
over, you will be required to present on your model. Part of your score
will be determine by how well your model performs and extra credit will
be given to the top scoring individuals.

\subsubsection{Project}\label{project}

During the latter portion of the course, you will complete a final
project that involves a deep exploration of a problem. More details for
the final project will be provided later in the course.

\subsection{Grading}\label{grading}

The final course grade will be calculated as follows:

\begin{longtable}[]{@{}ll@{}}
\toprule\noalign{}
Category & Percentage \\
\midrule\noalign{}
\endhead
\bottomrule\noalign{}
\endlastfoot
Homework & 10\% \\
Job Application 1 & 15\% \\
Job Application 2 & 15\% \\
Job Interview 1 & 15\% \\
Job Interview 2 & 15\% \\
Hack-a-thon \& Presentation & 15\% \\
Final Project & 15\% \\
\end{longtable}

The final letter grade will be determined based on the following
thresholds:

\begin{longtable}[]{@{}ll@{}}
\toprule\noalign{}
Letter Grade & Final Course Grade \\
\midrule\noalign{}
\endhead
\bottomrule\noalign{}
\endlastfoot
A & \textgreater= 93 \\
A- & 90 - 92.99 \\
B+ & 87 - 89.99 \\
B & 83 - 86.99 \\
B- & 80 - 82.99 \\
C+ & 77 - 79.99 \\
C & 73 - 76.99 \\
C- & 70 - 72.99 \\
D+ & 67 - 69.99 \\
D & 63 - 66.99 \\
D- & 60 - 62.99 \\
F & \textless{} 60 \\
\end{longtable}

\subsection{Course policies}\label{course-policies}

\subsubsection{Academic honesty}\label{academic-honesty}

\textbf{TL;DR: Don't cheat!}

\begin{itemize}
\item
  The job application assignments must be completed individually but you
  are welcome to discuss the assignment with classmates (e.g., discuss
  what's the best way for approaching a problem, what functions are
  useful for accomplishing a particular task, etc.). However you may not
  directly share (i.e.~via copy/paste or copying) any code or prose with
  anyone other than myself.
\item
  For the hack-a-thon, everyone will submit their predictions and give
  their own presentations. However, you are encouraged to work together.
  You are allowed to share code with one another. However, everyone
  should be able to explain what they did and everyone's projects should
  be unique in some way. Point reductions will be given if two
  individuals submit the exact same predictions.
\item
  For the projects, collaboration within teams is not only allowed, but
  expected. Communication between teams at a high level is also allowed
  however you may not share code or components of the project across
  teams.
\item
  \textbf{Reusing code}: Unless explicitly stated otherwise, you may
  make use of online resources (e.g.~StackOverflow) for coding examples
  on assignments. If you directly use code from an outside source (or
  use it as inspiration), you must explicitly cite where you obtained
  the code. Any recycled code that is discovered and is not explicitly
  cited will be treated as plagiarism.
\item
  \textbf{Use of artificial intelligence (AI)}: You should treat AI
  tools, such as ChatGPT, the same as other online resources. There are
  two guiding principles that govern how you can use AI in this
  course:\footnote{These guiding principles are based on
    \href{https://docs.google.com/document/d/1WpCeTyiWCPQ9MNCsFeKMDQLSTsg1oKfNIH6MzoSFXqQ/preview}{\emph{Course
    Policies related to ChatGPT and other AI Tools}} developed by Joel
    Gladd,
    Ph.D.\href{https://sta101-f23.github.io/course-syllabus.html\#fnref1}{↩︎}}
  (1) \emph{Cognitive dimension:} Working with AI should not reduce your
  ability to think clearly. We will practice using AI to
  facilitate---rather than hinder---learning. (2) \emph{Ethical
  dimension}\textbf{:} Students using AI should be transparent about
  their use and make sure it aligns with academic integrity. In general
  if the following two things are not true, you are cheating:

  \begin{itemize}
  \item
    You understand and can explain all of the code you have written down
    or you don't and you have cited the source of that code.
  \item
    All of your prose and narrative were written by yourself.
  \end{itemize}
\end{itemize}

If you are unsure if the use of a particular resource complies with the
academic honesty policy, just ask.

Regardless of course delivery format, it is the responsibility of all
students to understand and follow all College of Idaho policies,
including academic integrity (e.g., completing one's own work, following
proper citation of sources, adhering to guidance around group work
projects, and more). Ignoring these requirements is a violation of the
Honor Code.

\subsubsection{Late work policy}\label{late-work-policy}

The due dates for assignments are there to help you keep up with the
course material and to ensure the teaching team can provide feedback
within a timely manner. I understand that things come up periodically
that could make it difficult to submit an assignment by the deadline.

\begin{itemize}
\item
  \textbf{Late Homework:} Homework is completion based and well be
  accepted without penalty for a week. However, if your homework is
  turned in after I begin grading it, you will not receive any feedback.
\item
  \textbf{School-Sponsored Events/Illness:} If an assignment or meeting
  must be missed due to a school-sponsored event, you must let me know
  at least a week ahead of time so that we can schedule a time for you
  to make up the work before you leave. If an assignment or meeting must
  be missed due to illness, you must let me know as soon as it is safe
  for you to do so and before the assignment or meeting if possible.
  Failure to adhere to this policy will result in a 35\% penalty on the
  corresponding assignment.
\end{itemize}




\end{document}
